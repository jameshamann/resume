% XeLaTeX can use any Mac OS X font. See the setromanfont command below.
% Input to XeLaTeX is full Unicode, so Unicode characters can be typed directly into the source.

% The next lines tell TeXShop to typeset with xelatex, and to open and save the source with Unicode encoding.

%!TEX TS-program = xelatex
%!TEX encoding = UTF-8 Unicode

% \documentclass[12pt]{article}
% \usepackage{geometry}                % See geometry.pdf to learn the layout options. There are lots.
% \geometry{letterpaper}                   % ... or a4paper or a5paper or ... 
% %\geometry{landscape}                % Activate for for rotated page geometry
% %\usepackage[parfill]{parskip}    % Activate to begin paragraphs with an empty line rather than an indent
% \usepackage{graphicx}
% \usepackage{amssymb}

% Will Robertson's fontspec.sty can be used to simplify font choices.
% To experiment, open /Applications/Font Book to examine the fonts provided on Mac OS X,
% and change "Hoefler Text" to any of these choices.

% \usepackage{fontspec,xltxtra,xunicode}
% \defaultfontfeatures{Mapping=tex-text}
% \setromanfont[Mapping=tex-text]{Hoefler Text}
% \setsansfont[Scale=MatchLowercase,Mapping=tex-text]{Gill Sans}
% \setmonofont[Scale=MatchLowercase]{Andale Mono}

% \title{Brief Article}
% \author{The Author}
% %\date{}                                           % Activate to display a given date or no date

% \begin{document}
% \maketitle

% For many users, the previous commands will be enough.
% If you want to directly input Unicode, add an Input Menu or Keyboard to the menu bar 
% using the International Panel in System Preferences.
% Unicode must be typeset using a font containing the appropriate characters.
% Remove the comment signs below for examples.

% \newfontfamily{\A}{Geeza Pro}
% \newfontfamily{\H}[Scale=0.9]{Lucida Grande}
% \newfontfamily{\J}[Scale=0.85]{Osaka}

% Here are some multilingual Unicode fonts: this is Arabic text: {\A السلام عليكم}, this is Hebrew: {\H שלום}, 
% and here's some Japanese: {\J 今日は}.



% \end{document}  %!TEX TS-program = xelatex
%!TEX encoding = UTF-8 Unicode
% Awesome CV LaTeX Template
%
% This template has been downloaded from:
% https://github.com/posquit0/Awesome-CV
%
% Author:
% Claud D. Park <posquit0.bj@gmail.com>
% http://www.posquit0.com
%
% Template license:
% CC BY-SA 4.0 (https://creativecommons.org/licenses/by-sa/4.0/)
%


%%%%%%%%%%%%%%%%%%%%%%%%%%%%%%%%%%%%%%
%     Configuration
%%%%%%%%%%%%%%%%%%%%%%%%%%%%%%%%%%%%%%
%%% Themes: Awesome-CV
\documentclass[]{awesome-cv}
\usepackage{textcomp}
%%% Override a directory location for fonts(default: 'fonts/')
\fontdir[fonts/]

%%% Configure a directory location for sections
\newcommand*{\sectiondir}{resume/}

%%% Override color
% Awesome Colors: awesome-emerald, awesome-skyblue, awesome-red, awesome-pink, awesome-orange
%                 awesome-nephritis, awesome-concrete, awesome-darknight
%% Color for highlight
% Define your custom color if you don't like awesome colors
\colorlet{awesome}{awesome-red}
%\definecolor{awesome}{HTML}{CA63A8}
%% Colors for text
%\definecolor{darktext}{HTML}{414141}
%\definecolor{text}{HTML}{414141}
%\definecolor{graytext}{HTML}{414141}
%\definecolor{lighttext}{HTML}{414141}

%%% Override a separator for social informations in header(default: ' | ')
%\headersocialsep[\quad\textbar\quad]
\begin{document}
    
%%%%%%%%%%%%%%%%%%%%%%%%%%%%%%%%%%%%%%
%     Profile
%%%%%%%%%%%%%%%%%%%%%%%%%%%%%%%%%%%%%%
\begin{center}
	\headerfirstnamestyle{James} \headerlastnamestyle{Hamann} \\
	\vspace{2mm}
	{\faEnvelope\ jameshamann0@gmail.com} | {\faMapMarker\ Greater London, UK} | {\faLink\ jameshamann.com}
\end{center}
%%%%%%%%%%%%%%%%%%%%%%%%%%%%%%%%%%%%%%
%     Experience
%%%%%%%%%%%%%%%%%%%%%%%%%%%%%%%%%%%%%%
\cvsection{Experience}
\begin{cventries}
	\cventry
	{}
	{Personal Profile}
	{}
	{}
	{\begin{cvitems}
		\item {I am a highly motivated and enthusiastic software developer with a passion for creating innovative solutions. I have a strong background in Ruby on Rails and JavaScript, with some experience in Python.I have a proven track record in developing and maintaining applications in a variety of environments.}
		\item {I graduated with a 2:1 in June 2015 after completing a BSc in Music Technology. During my studies, I discovered a passion for programming and development, which led me to apply for a place at Makers Academy, an intensive coding bootcamp in London. At Makers Academy, I built a strong foundation in core programming principles and secured my first development job shortly after graduating.}
		\item {In my current role, I handle a variety of development and maintenance tasks and lead the business’s system goals. I excel at understanding client requirements and translating them into effective features and solutions. My technical skills include proficiency in several programming languages, frameworks, and tools, which I have utilised in both professional and personal projects.}
		\item {Outside of work, I enjoy starting side projects to explore different technologies. Currently, I am working on Recikeep, a React Native application that serves as a digital recipe book. In addition, I write about tech on my Medium blog and try to keep up to date with what\textquotesingle{}s happening across the industry.}
		\end{cvitems}}
	\cventry
	{Lead Software Developer}
	{Sample Logistics}
	{Gatwick, UK}
	{Jan 2022 – Present}
	{\begin{cvitems}
		\item {Managing timelines, milestones and deliverables ensuring our goals are met as a company.}
		\item {Taking business requirements and client requests and translating into solutions that we could develop.}
		\item {Leading our API Integrations with both clients and suppliers ensuring each system is well-documented and scalable.}
		\item {Promoting our client-facing systems to new clients and analysing their requirements to provide the best solution available.}
		\item {Implemented a series of microservices and migrated some older applications to a new architecture style, leveraging AWS to develop features faster and troubleshoot more effectively.}
		\item {Continually improving existing applications through adding new features and diagnosing and fixing existing issues.}
		\end{cvitems}}
	\cventry
	{Software Developer}
	{Sample Logistics}
	{Gatwick, UK}
	{Dec 2016 – Dec 2021}
	{\begin{cvitems}
		\item {Assisted in migrating applications from an on-premises server setup to AWS, setting up EC2 instances with PostgreSQL, and managing domains.}
		\item {Developed an internal portal application for the business to improve traceability of shipments and provide real-time tracking for users.}
		\item {Maintained and developed new features for our client-facing applications.}
		\item {Managed custom API integrations with our suppliers to facilitate shipments in new locations.}
		\end{cvitems}}
	\cventry
	{New Business Development Executive}
	{Customer Perspectives}
	{Horsham, UK}
	{Jul 2015 – Jul 2016}
	{\begin{cvitems}
		\item {Researching and generating new leads for the company to gain new business.}
		\item {Setting up and leading meetings with clients, explaining the business services and our key USPs.}
		\end{cvitems}}
\end{cventries}
\cvsection{Skills}
\begin{cventries}
	\cventry
	{}
	{\def\arraystretch{1.15}{\begin{tabular}{ l l }
		Ruby:  & {\skill{ Ruby on Rails, API Development, MVC, Web Scraping}} \\
		JavaScript:  & {\skill{ React, React Native, TypeScript, Node.js}} \\
		Python:  & {\skill{ Data Analysis, Database Scripts, Web Scraping, Migration Scripts, IoT (RaspberryPI)}} \\
		AWS:  & {\skill{ Elastic Beanstalk, Lambda, S3, Cognito, Amplify, RDS, DynamoDB}} \\
		\end{tabular}}}
	{}
	{}
	{}
\end{cventries}

\vspace{-7mm}
%%%%%%%%%%%%%%%%%%%%%%%%%%%%%%%%%%%%%%
%     Education
%%%%%%%%%%%%%%%%%%%%%%%%%%%%%%%%%%%%%%
\cvsection{Education}
\begin{cventries}
	\cventry
	{BSc in Music Technology}
	{University of Kent}
	{Kent, UK}
	{Sept 2012 – Jun 2015}
	{}
	\cventry
	{Software Development Bootcamp}
	{Makers Academy}
	{London, UK}
	{Aug 2016 – Nov 2016}
	{}
\end{cventries}

\vspace{-2mm}
\vspace{28mm}
\cvsection{Projects}
\begin{cventries}
	\cventry
	{A React Native Application for keen cooks and chefs, functioning as a recipe book allowing users to save their recipes and explore new recipes.}
	{Recikeep}
	{React Native, GraphQL, DynamoDB}
	{}
	{}
	
	\vspace{-5mm}
	\cventry
	{A Chrome plug-in that displays live cryptocurrency rates in USD, GBP, and EUR.}
	{CrypCheck}
	{JavaScript}
	{}
	{}
	
	\vspace{-5mm}
	\cventry
	{A Ruby Gem for use with the Static Site Generator Jekyll. The theme follows Google\textquotesingle{}s material design principals, making use of Materialize CSS. 25k Downloads to date.}
	{Jekyll Material Theme}
	{Ruby}
	{}
	{}
	
	\vspace{-5mm}
	\cventry
	{A Ruby Gem that fetches your RSS feed, parses and then saves the request to be used within your Jekyll site. 4k Downloads to date.}
	{Jekyll Display Medium Blog Posts}
	{Ruby}
	{}
	{}

	\vspace{-5mm}
	\cventry
	{An iOS location based app developed as part of our final project during Makers Academy}
	{Turfy}
	{Swift, Xcode, Firebase}
	{}
	{}

	\vspace{-5mm}
	\cventry
	{A Node.js App for RaspberryPI that serves as a Smart Mirror displaying the time, weather, and news as well as basic Amazon Alexa integration.}
	{Miri}
	{JavaScript, React, Node.js, RaspberryPI}
	{}
	{}
	{}
	{}

	\vspace{-5mm}
	\cventry
	{My personal blog where I write about tech, development, and other topics.}
	{Medium Blog}
	{Writing, Tech, Development}
	{https://medium.com/@jameshamann}
	{}
	{}
	{}
	{}
	{}
	{}
	
	\vspace{-5mm}
\end{cventries}

\ 
\end{document}